\begin{tcolorbox}
Die Produktfunktionen beschreiben jede einzelne Funktion des Produkts mittels Anwendungsfalldiagrammen und Anwendungsfalltabellen.
Diese sollen möglichst ausschlaggebend für das zu entwickelnde System sein und nicht simple Produktfunktionen wie z.B. Login, Account erstellen, Gruppe beitreten, Passwort ändern oder ähnliches zeigen.
\autoref{fig:anwendungsfall-app-tabelle-xx-1} stellt eine exemplarische Tabelle für die Beschreibung eines Anwendungsfalls dar. Stil und Formatierung sind variabel. Nicht jede Zelle muss immer gefüllt sein.
\\\\
In  Tabelle~\autoref{fig:akteur-tabelle} werden alle auftretenden Akteure beschrieben.


\end{tcolorbox}

\begin{figure}[h]
	\centering
	
	\begin{tabularx}{\textwidth}{ p{.2\textwidth} | p{.2\textwidth} | X }
		\textbf{Akteur} & \textbf{Beschreibung} & \textbf{Verwendet in Anwendungsfall} \\ \hline
		Informatiker & Programmiert tolle Sachen & Programmieren, Kaffee trinken, Schlafen
	\end{tabularx}
	
	\caption{Beschreibung der Akteure}
	\label{fig:akteur-tabelle}
\end{figure}


%%%%%%%%%%%%%%%
%% Anwendungsfall 1 %%
%%%%%%%%%%%%%%%

\section{Anwendungsfalldiagramm - App}

\begin{figure}[h]
	\centering
	\missingfigure{Anwendungsfalldiagramm - App}		
	\caption{Anwendungsfalldiagramm - App}
	\label{fig:anwendungsfalldiagramm-app}
\end{figure}

\newpage

\begin{figure}[h]
	\centering
	\begin{tabularx}{\textwidth}{ X | X }
		\textbf{Anwendungsfall ID} & XX-1 \\ \hline
		\textbf{Anwendungsfallname} & Hier steht ein Name. \\ \hline
		\textbf{Initiierender Akteur} & Informatiker \\ \hline
		\textbf{Weitere Akteure} & Designer, Techniker  \\ \hline
		\textbf{Kurzbeschreibung} & Hier steht eine Kurzbeschreibung.  \\ \hline
		\textbf{Vorbedingungen} & -  \\ \hline
		\textbf{Nachbedingungen} & Y trifft zu.  \\ \hline
		\textbf{Ablauf} &
			\begin{enumerate}
				\item Erster ganzer Satz.
				\item Zweiter ganzer Satz.
			\end{enumerate} \\ \hline
		\textbf{Alternative} &
				\begin{enumerate}
					\item Erster ganzer Satz.
					\item Zweiter ganzer Satz.
				\end{enumerate}  \\ \hline
		\textbf{Ausnahme} &
				\begin{enumerate}
					\item Erster ganzer Satz.
					\item Zweiter ganzer Satz.
				\end{enumerate}  \\ \hline
		\textbf{Benutzte Anwendungsfälle} & YY-1 (oder Name) \\ \hline
		\textbf{Spezielle Anforderungen} & - \\ \hline
		\textbf{Annahmen} & -
	\end{tabularx}
	\caption{Anwendungsfall XX-1}
	\label{fig:anwendungsfall-app-tabelle-xx-1}
\end{figure}

\newpage


%%%%%%%%%%%%%%%
%% Anwendungsfall 2 %%
%%%%%%%%%%%%%%%

\section{Anwendungsfalldiagramm - Server}

\begin{figure}[h]
	\centering
	\missingfigure{Anwendungsfalldiagramm - Server}		
	\caption{Anwendungsfalldiagramm - Server}
	\label{fig:anwendungsfalldiagramm-server}
\end{figure}

\newpage

\begin{figure}[h]
	\centering
	\begin{tabularx}{\textwidth}{ X | X }
		\textbf{Anwendungsfall ID} & XX-1 \\ \hline
		\textbf{Anwendungsfallname} & Hier steht ein Name. \\ \hline
		\textbf{Initiierender Akteur} & Informatiker \\ \hline
		\textbf{Weitere Akteure} & Designer, Techniker  \\ \hline
		\textbf{Kurzbeschreibung} & Hier steht eine Kurzbeschreibung.  \\ \hline
		\textbf{Vorbedingungen} & -  \\ \hline
		\textbf{Nachbedingungen} & Y trifft zu.  \\ \hline
		\textbf{Ablauf} &
		\begin{enumerate}
			\item Erster ganzer Satz.
			\item Zweiter ganzer Satz.
		\end{enumerate} \\ \hline
		\textbf{Alternative} &
		\begin{enumerate}
			\item Erster ganzer Satz.
			\item Zweiter ganzer Satz.
		\end{enumerate}  \\ \hline
		\textbf{Ausnahme} &
		\begin{enumerate}
			\item Erster ganzer Satz.
			\item Zweiter ganzer Satz.
		\end{enumerate}  \\ \hline
		\textbf{Benutzte Anwendungsfälle} & YY-1 (oder Name) \\ \hline
		\textbf{Spezielle Anforderungen} & - \\ \hline
		\textbf{Annahmen} & -
	\end{tabularx}
	\caption{Anwendungsfall XX-1}
	\label{fig:anwendungsfall-server-tabelle-xx-1}
\end{figure}

%%%%%%%%%%%%%%%
%% Abgrenzungskriterien %%
%%%%%%%%%%%%%%%

\section{Abgrenzungskriterien}

Nur das letzte Bild wird für User sichtbar sein, die letzten 10 für Admins.
Die App wird keine Verträge anzeigen oder ändern können.
Die Kunden werden nicht über die App bezahlen können.
Wir, das Entwicklerteam, können nicht für die Richtigkeit der von den Kunden hochgeladenen Zählerstände garantieren.
Ein Kunde wird nur genau ein Kundenkonto erhalten können.
Kunden werden sich nicht selbst registrieren können. Admins müssen für jeden Kunden ein Konto manuell erstellen.
Kunden können Bilder von Messständen nur über die App hochladen und nicht über die Website.

