\textbf{Anmerkung:} \\
Endpunkte, welche ein Paging\textless T\textgreater \  Objekt zurück geben sind pageable und sortierbar. Dafür können in der Request weitere Parameter angeben werden. Allerdings sind wir uns noch nicht sicher wie diese Parameter, in den von Spring angebotenen Libraries, aussehen. Deshalb haben wir diese Parameter vorerst weggelassen. Weiter sind noch weitere Status Codes als Return Wert möglich um Fehler zu handeln. Zum Beispiel ist an allen Endpunkten eine Authentifizierung nötig, sodass 401 ein weiterer möglicher Return Code ist. Alternative Fehlercode haben wir vorerst aber noch nicht genau definiert.  \\ \\ \\
 

\begin{lstlisting}
interface Paging<T> {
	content: T[]
	size: number
	page: number
	totalPages: number
	total: number
	first: boolean
	last: boolean
}
\end{lstlisting}


\begin{lstlisting}
interface MeterDTO {
	id: string
	type: "water" | "eletricity" | "gas"
	name: string
	ownerId: string | null
	lastReading: ReadingDTO
	meterNumber: string
	createdAt: UTC string
	updatedAt: UTC string | null
	deletedAt: UTC string | null
}
\end{lstlisting} \newpage

\begin{lstlisting}
interface ReadingDTO {
	id: string
	meterId: string
	ownerId: string
	value: string
	trend: number
	lastEditorName: string
	lastEditReason: string
	createdAt: UTC string
	updatedAt: UTC string | null
	deletedAt: UTC string | null
}
\end{lstlisting}

\begin{lstlisting}
interface IssueDTO {
	id: string
	email: string
	name: string
	subject: string
	message: string
	status: "UNRESOLVED" | "RESOLVED"
	createdAt: UTC string
	updatedAt: UTC string | null
	deletedAt: UTC string | null
}
\end{lstlisting}


\begin{lstlisting}
interface UserDTO {
	id: string
	customerId: string
	firstName: string
	lastName: string
	email: string
	createdAt: UTC string
	updatedAt: UTC string | null
	deletedAt: UTC string | null
}
\end{lstlisting}

\begin{lstlisting}
interface PictureResultDTO {
	meterId: string
	value: string
}
\end{lstlisting}

\begin{figure}[H]
	\textbf {Zähler} \\ \\
	\begin{tabularx}{\textwidth}{X | X | X}
		\hline \\
		GET 	/api/meters & Liste aller Zähler & 200 : Paging \textless MeterDTO\textgreater \\ \\ \hline
		\\ POST /api/meters & Zähler erstellen & 201 : MeterDTO\\ \\ \hline
		\\  PUT /api/meters/\{mid\} & Zähler updaten & 200 : MeterDTO \\ \\ \hline
		\\ DELETE	/api/meters/\{mid\} & Zähler löschen & 204 \\ \\ \hline
	\end{tabularx}
\\ \\ \\ \\

	\textbf{Zählerstände} \\ \\
	\begin{tabularx}{\textwidth}{X | X | X}
		\hline \\
		GET	/api/meters/\{mid\} /readings & Alle Zählerstände zu einem Zähler & 200 : Paging \textless UserDTO \textgreater \\ \\ \hline
	\\ 	POST /api/meters/\{mid\} /readings & Neuen Zählerstand einem Zähler hinzufügen & 201 : ReadingDTO\\ \\ \hline
	\\	PUT /api/meters/\{mid\} /readings/\{rid\} & Einen Zählerstand eines Zählers updaten & 200 : ReadingDTO \\ \\ \hline
	\\	DELETE /api/meters/\{mid\} /readings/\{rid\} & Einen Zählerstand eines Zählers entfernen & 204 \\ \\   \hline
	\\	POST /api/picutre & Zählerstand + Zähler vom Bild erkennen & 200 : PictureResultDTO \\ \\ \hline
	\end{tabularx}

\end{figure}

\begin{figure}[H]
	\textbf{Benutzer}\\ \\
	\begin{tabularx}{\textwidth}{X | X | X}
	\hline \\
	GET /api/users & Liste aller Benutzer & 200 : Paging \textless UserDTO \textgreater\\ \\ \hline
	\\ GET /api/users/\{uid\} & Infos zu einem Benutzer & 200 : UserDTO \\ \\ \hline
	\\ GET /api/users/\{uid\}/meters & Liste aller Zähler zu einem Benutzer & 201 : Paging \textless MeterDTO \textgreater  \\ \\ \hline
	\\ GET /api/users/\{uid\}/meters/\{mid\} /readings & Liste aller Zählerstände für einen Zähler von einem Benutzer &  200 : Paging \textless ReadingDTO \textgreater \\ \\ \hline
	\\ POST /api/users & Benutzer hinzufügen &201 : UserDTO \\ \\ \hline
	\\ PUT /api/users/\{uid\} & Benutzer updaten & 200 : UserDTO\\ \\ \hline
	\\ DELETE /api/users/\{uid\} & Benutzer löschen & 204\\ \\ \hline
	\end{tabularx}
\end{figure}

\begin{figure}[H]
	\textbf{Ticket}\\ \\
	\begin{tabularx}{\textwidth}{X | X | X}
	\hline \\
	GET /api/issues & Liste aller Tickets & 201 : Paging \textless IssueDTO \textgreater\\ \\ \hline
	\\ POST /api/issues/\{iid\} & Ticket hinzufügen & 201 : IssueDTO \\ \\ \hline
	\\ PUT /api/issues/\{iid\} & Updaten eines Tickets & 200 : IssueDTO \\ \\ \hline
	\\ DELETE /api/issues/\{iid\} & Ticket löschen & 204 \\ \\ \hline
	
	\end{tabularx}
\end{figure}

\newpage

\begin{figure}[H]
	\textbf{Authentifizierung}\\ \\
	\begin{tabularx}{\textwidth}{X | X | X}
	\hline \\
	POST /api/login & Login Token für Benutzer erhalten & 200 : JWT Token \\ \\ \hline
	\\ GET /api/logout & Benutzer ausloggen & 204 \\ \\ \hline
	\end{tabularx}
\end{figure}






