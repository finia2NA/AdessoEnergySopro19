\begin{tcolorbox}
In diesem Glossar können Akronyme und abkürzende Schreibweisen aufgelistet werden. 
Alle verwendeten Abkürzungen innerhalb des Projekts müssen hier erläutert werden.
\end{tcolorbox}

\begin{table}[h]
	\centering
	\begin{tabularx}{\textwidth}{X X}
		\rowcolor[HTML]{C0C0C0} 
		\textbf{Abkürzung} & \textbf{Beschreibung} \\
		Zähler & Bei einem Zähler handelt es sich um einen Gas-, Strom- oder Wasserzähler. Er misst den Verbrauch der jeweils namensgebenden Ressource. \\
		\rowcolor[HTML]{E7E7E7} 
		FAB (Floating Action Button) & Ein FAB ist ein Knopf, der in der unteren rechten Ecke einer Android-App sitzt und Zugriff zu essentiellen Funktionen ermöglicht. \\
		Administrator (Admin) & Der Administrator, vornehmlich ein Mitarbeiter von adesso energy, hat übergeordnete Zugriffsrechte, die es ihm erlauben, auf alle Daten zuzugreifen, sowie diese zu ändern. Er interagiert hierfür ausschließlich mit der Website. \\
		\rowcolor[HTML]{E7E7E7} 
		Benutzer (User) & Als Benutzer werden Kunden von adesso energy bezeichnet. Sie haben die Möglichkeit, über die App oder Website auf ihre Nutzerdaten zuzugreifen und aktuelle Zählerstände hochzuladen. Dabei bietet die App die Option, einen Zählerstand automatisch aus einem Bild zu erkennen. Der Benutzer kann sowohl mit der App als auch mit der Website interagieren. \\
		Web-Applikation & Der Server ist nicht Teil einer Web-Applikation. Aufgrund der umfassenden React-Architektur, betrachten wir die Website als eigene Applikation. \\
		\rowcolor[HTML]{E7E7E7} 
		DTO (data transfer object) & Als data transfer objects bezeichnen wir die Objekte, die unsere Anwendungen über das HTTP-Protokoll übertragen.
	\end{tabularx}
	\caption{Glossar}
	\label{table:glossar}
\end{table}