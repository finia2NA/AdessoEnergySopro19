Die folgende Aufzählung beschreibt die Attribute verschiedener Datensätze und ihre jeweiligen Beziehungen zueinander.

\subsubsection*{Benutzer}
	\begin{itemize}
	    \item \textbf{Kunden-ID}\hfill\\
	    Jeder Benutzer bekommt von unserem System eine eindeutige ID, welche als Schlüssel fungiert. Dies passiert, um möglich Fehler aus dem System des Kunden nicht mit zu übernehmen.
  		\item \textbf{Vorname}\hfill \\
  		Jede Benutzer hat mindestens einen Vornamen.
 		\item \textbf{Nachname}\hfill \\
 		Jeder Benutzer hat mindestens einen Nachnamen.
 		\item \textbf{Kundennummer}\hfill \\
 		Eine Benutzer hat vom Anbieter bereitgestellte Kundennummer.
 		\item \textbf{Passwort} \hfill \\
 		Jede Benutzer hat ein Passwort.
		\item \textbf{Zähler}\hfill \\
		Jeder Benutzer werden eine Menge an Zählern zugeordnet.
		\item \textbf{e-Mail Adresse}\hfill \\
		Mit einer Benutzer wird eine e-Mail Adresse assoziiert.
		\item \textbf{created at}\hfill \\
		Das Datum an dem der Eintrag angelegt wurde.
		\item \textbf{deleted at}\hfill \\
		Das Datum an dem eine Benutzer den Vertrag mit der Firma gekündigt hat. Ist null, falls die Benutzer einen laufenden Vertrag hat.
		\item \textbf{updated at}\hfill \\
		Das Datum der letzten Änderung an dem Eintrag der Benutzer.
	\end{itemize}
\subsubsection*{Zähler}
	\begin{itemize}
	    \item \textbf{Zähler-ID}\hfill\\
	    Jedem Zähler wird eine eindeutige ID zugewiesen.
		\item \textbf{Art}\hfill \\
		Ein Zähler ist entweder ein Gas-, Strom- oder Wasserzähler.
		\item \textbf{Zählernummer}\hfill \\
		Ein Zähler hat eine Zählernummer. Diese hat verschiedene Formate für Gas-, Strom- und Wasserzähler.
		\item \textbf{created at}\hfill \\
		Das Datum an dem der Eintrag angelegt wurde.
		\item \textbf{deleted at}\hfill \\
		Das Datum an dem ein Zähler entfernt wurde. Ist null, falls der Zähler existiert.
		\item \textbf{updated at}\hfill \\
		Das Datum der letzten Änderung des Eintrages des Zählers.
		\item \textbf{Zählerstand}\hfill\\
		Jedem Zähler wird eine Menge von Zählerständen zugeordnet.
	\end{itemize}
\subsubsection*{Stand}
	\begin{itemize}
	    \item \textbf{Stand-ID}\hfill\\
	    Jedem Stand wird eine eindeutige ID zugeordnet.
		\item \textbf{Wert}\hfill \\
		Ein Zählerstand hat eine Zahl, die den Wert des Zählerstands repräsentiert.
		\item \textbf{created at}\hfill \\
		Das Datum, an dem der Stand gemessen wurde.
		\item \textbf{updated at}\hfill \\
		Das Datum, an dem der Stand das letzte Mal geändert wurde. Wird nur genutzt sofern Korrekturen durchgeführt wurden.
	\end{itemize}
\subsubsection*{Adresse}
	\begin{itemize}
	    \item \textbf{Adressen-ID}\hfill\\
	    Jeder Adresse wird eine eindeutige ID zugeordnet.
		\item \textbf{Straßenname}\hfill\\
		\item \textbf{PLZ} \hfill\\		
		\item \textbf{Hausnummer}\hfill\\
	\end{itemize}
\subsubsection*{Kontaktformular}
	\begin{itemize}
	    \item \textbf{Nachrichten-ID}\hfill\\
	    Jeder Nachricht die über das Kontaktformular gesendet wird, wird eine eindeutige ID zugeordnet.
		\item Name \hfill \\
		Der Name mit dem der Benutzer während des Mailverkehrs angesprochen werden möchte.
		\item e-Mail Adresse \hfill \\
		Die Mailadresse über die der Administrator den Benutzer bezüglich seines Problems kontaktiert. 
		\item Betreff \hfill \\
		Der Betreff der Nachricht des Benutzers.
		\item Nachricht \hfill \\
		Die Nachricht in der der Benutzer sein Problem schildert.
	\end{itemize}	
	
