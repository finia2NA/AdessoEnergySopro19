\begin{tcolorbox}
Die Produktfunktionen beschreiben jede einzelne Funktion des Produkts mittels Anwendungsfalldiagrammen und Anwendungsfalltabellen.
Diese sollen möglichst ausschlaggebend für das zu entwickelnde System sein und nicht simple Produktfunktionen wie z.B. Login, Account erstellen, Gruppe beitreten, Passwort ändern oder ähnliches zeigen.
\autoref{fig:anwendungsfall-app-tabelle-xx-1} stellt eine exemplarische Tabelle für die Beschreibung eines Anwendungsfalls dar. Stil und Formatierung sind variabel. Nicht jede Zelle muss immer gefüllt sein.
\\\\
In  Tabelle~\autoref{fig:akteur-tabelle} werden alle auftretenden Akteure beschrieben.


\end{tcolorbox}

\begin{figure}[h]
	\centering
	
	\begin{tabularx}{\textwidth}{ p{.2\textwidth} | p{.2\textwidth} | X }
		\textbf{Akteur} & \textbf{Beschreibung} & \textbf{Verwendet in Anwendungsfall} \\ \hline
		Informatiker & Programmiert tolle Sachen & Programmieren, Kaffee trinken, Schlafen
	\end{tabularx}
	
	\caption{Beschreibung der Akteure}
	\label{fig:akteur-tabelle}
\end{figure}

%%%%%%%%%%%%%%%
%% Eigene Arbeit %%
%%%%%%%%%%%%%%%
\section{Features}
Das folgende Kapitel behandelt alle erdachten Features und unterteilt diese in die Kategorien Muss-, Soll- oder Kann-Features. 
Die Muss-Features sind dabei essentiell für die Funktionalität der Software und haben höchste Priorität.
Soll-Features sind Erweiterungen der Grundfunktionen oder Verbesserungen der Muss-Features. Dabei ist die Grundfunktionalität der Software bereits durch die Muss-Features abgedeckt.
Wenn genug Zeit vorhanden ist, dann werden Funktionalitäten aus der Kategorie der Kann-Features implementiert. 
Diese stellen eine sinnvolle Erweiterung zum Projekt dar, sind aber im Gegensatz zu den Soll-Features nicht Teil der ursprünglichen Anforderungen.
\subsection{Muss-Features}
Die folgenden Features sind von uns als grundlegend eingestuft worden und umfassen die Kernfunktionalitäten von App und Anwendung.
\begin{itemize}
\item \textbf{History} \hfill \\
	In der App und auf der Website lassen sich die letzten Zählerstände für einen Account abrufen.
	Dabei wird der letzte Stand als Bild und weitere als Wert angezeigt.
\item \textbf{Scannen} \hfill \\
	In der App können Fotos aufgenommen und anschließend hochgeladen werden. 
	Wird von Azure ein Zähler auf dem Foto erkannt, dann wird der Zählerstand ausgelesen und an die Datenbank übermittelt.
	Außerdem wird das Foto gespeichert.
\item \textbf{Manuelles Eintragen} \hfill \\
	Auf der Website und in der App lassen sich Zählerart und Zählerstände manuell eintragen.
\item \textbf{Nutzerverwaltung} \hfill \\
	Mit Administrationsrechten können Kund*innendaten eingesehen, verändert oder neu ins System eingetragen werden.
	Dabei lassen sich Ergebnisse sortieren und filtern.
\end{itemize}

\subsection{Soll-Features}
Die Soll-Features sind Erweiterungen der Grundfunktionen oder stellen Verbesserungen bzw. Änderungen an diesen dar.
\begin{itemize}
\item \textbf{Kontaktformular} \hfill \\
	Nutzende können über ein Kontaktformular in der App oder auf der Website Anfragen schicken.
	In der Ansicht der Administrierenden können diese dann angesehen und bearbeitet werden. 
\item \textbf{Bild aus Galerie} \hfill \\
	Anstatt Bilder zum Auswerten des Zählerstandes in der App aufzunehmen, ist es auch möglich Bilder aus der Galerie des mobilen Endgerätes auszuwählen.
\item \textbf{Push-Nachrichten} \hfill \\
	Die App kann Benachrichtigungen schicken, um an das Eintragen von Zählerständen zu erinnern.
	Diese lassen sich bei Bedarf ausschalten.
\item \textbf{Mehrere Zähler}\hfill \\
	Einem Account können mehrere Zähler gleicher Art zugeordnet werden.
\item \textbf{Fehlerbehandlung} \hfill \\
	Falls ein falsches oder unleserliches Bild hochgeladen wird, erkennt die App dies und ein neues Foto wird angefordert.
\item \textbf{Fehler in Zahlen erkennen} \hfill \\
	Im Backend wird der hochgeladene Zählerstand auf Plausibilität überprüft. Dies geschieht durch Vergleiche mit den vorigen Zählerständen. 
	Bei nicht plausiblem Zählerstand wird der Benutzer zu einer erneuten Bestätigung aufgefordert. 
\end{itemize}

\subsection{Kann-Features}
Die Kann-Features sind weiter Funktionalitäten die nicht zum (erweiterten)-Grundumfang gehören und stellen sinnvolle Erweiterungen des Projekts dar.
\begin{itemize}
\item  \textbf{Darkmode}
	In der App wird eine Dunkle Benutzeroberfläche zur verfügung gestellt.
\item \textbf{Sprachen}
	Die Sprache der Website und der App lässt sich auf englisch umstellen.
\item \textbf{Diagramm}
	Die History wird um eine grafische Darstellung erweitert, die die Zählerstände eines Zeitabschnitts anzeigt.
\item \textbf{Statistiken}
	Mit Administrationsrechten können die durchschnittlichen oder spezifische Verbrauchs- und Zählerstatistiken eingesehen werden.
\item \textbf{App Erkennung}
	Wenn die Website auf einem Endgerät genutzt wird, welches die App installiert hat, dann wird dies erkannt und
	die Nutzung der App vorgeschlagen.
\end{itemize}


%%%%%%%%%%%%%%%
%% Anwendungsfall 1 %%
%%%%%%%%%%%%%%%

\section{Anwendungsfalldiagramm - App}

\begin{figure}[h]
	\centering
	\missingfigure{Anwendungsfalldiagramm - App}		
	\caption{Anwendungsfalldiagramm - App}
	\label{fig:anwendungsfalldiagramm-app}
\end{figure}

\newpage

\begin{figure}[h]
	\centering
	\begin{tabularx}{\textwidth}{ X | X }
		\textbf{Anwendungsfall ID} & XX-1 \\ \hline
		\textbf{Anwendungsfallname} & Hier steht ein Name. \\ \hline
		\textbf{Initiierender Akteur} & Informatiker \\ \hline
		\textbf{Weitere Akteure} & Designer, Techniker  \\ \hline
		\textbf{Kurzbeschreibung} & Hier steht eine Kurzbeschreibung.  \\ \hline
		\textbf{Vorbedingungen} & -  \\ \hline
		\textbf{Nachbedingungen} & Y trifft zu.  \\ \hline
		\textbf{Ablauf} &
			\begin{enumerate}
				\item Erster ganzer Satz.
				\item Zweiter ganzer Satz.
			\end{enumerate} \\ \hline
		\textbf{Alternative} &
				\begin{enumerate}
					\item Erster ganzer Satz.
					\item Zweiter ganzer Satz.
				\end{enumerate}  \\ \hline
		\textbf{Ausnahme} &
				\begin{enumerate}
					\item Erster ganzer Satz.
					\item Zweiter ganzer Satz.
				\end{enumerate}  \\ \hline
		\textbf{Benutzte Anwendungsfälle} & YY-1 (oder Name) \\ \hline
		\textbf{Spezielle Anforderungen} & - \\ \hline
		\textbf{Annahmen} & -
	\end{tabularx}
	\caption{Anwendungsfall XX-1}
	\label{fig:anwendungsfall-app-tabelle-xx-1}
\end{figure}

\newpage


%%%%%%%%%%%%%%%
%% Anwendungsfall 2 %%
%%%%%%%%%%%%%%%

\section{Anwendungsfalldiagramm - Server}

\begin{figure}[h]
	\centering
	\missingfigure{Anwendungsfalldiagramm - Server}		
	\caption{Anwendungsfalldiagramm - Server}
	\label{fig:anwendungsfalldiagramm-server}
\end{figure}

\newpage

\begin{figure}[h]
	\centering
	\begin{tabularx}{\textwidth}{ X | X }
		\textbf{Anwendungsfall ID} & XX-1 \\ \hline
		\textbf{Anwendungsfallname} & Hier steht ein Name. \\ \hline
		\textbf{Initiierender Akteur} & Informatiker \\ \hline
		\textbf{Weitere Akteure} & Designer, Techniker  \\ \hline
		\textbf{Kurzbeschreibung} & Hier steht eine Kurzbeschreibung.  \\ \hline
		\textbf{Vorbedingungen} & -  \\ \hline
		\textbf{Nachbedingungen} & Y trifft zu.  \\ \hline
		\textbf{Ablauf} &
		\begin{enumerate}
			\item Erster ganzer Satz.
			\item Zweiter ganzer Satz.
		\end{enumerate} \\ \hline
		\textbf{Alternative} &
		\begin{enumerate}
			\item Erster ganzer Satz.
			\item Zweiter ganzer Satz.
		\end{enumerate}  \\ \hline
		\textbf{Ausnahme} &
		\begin{enumerate}
			\item Erster ganzer Satz.
			\item Zweiter ganzer Satz.
		\end{enumerate}  \\ \hline
		\textbf{Benutzte Anwendungsfälle} & YY-1 (oder Name) \\ \hline
		\textbf{Spezielle Anforderungen} & - \\ \hline
		\textbf{Annahmen} & -
	\end{tabularx}
	\caption{Anwendungsfall XX-1}
	\label{fig:anwendungsfall-server-tabelle-xx-1}
\end{figure}