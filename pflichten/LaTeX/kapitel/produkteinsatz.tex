Dieses Kapitel beschreibt die Rahmenbedingungen des Produkteinsatzes. Also die Anwendungsgebiete, die genauer bestimmen, was der Zweck des Produktes ist, sowie die Zielgruppe, für die das Produkt entwickelt wird,
und die Betriebsbedingungen, also die Umstände unter denen eine Firma das Produkt benutzen und warten kann.

\section{Anwendungsgebiete}
	Um den aktuellen Zählerstand an den Energieanbieter zu schicken mussten bisher die Stände abgelesen und per Post verschickt 		  
	werden. \\
	Das Produkt vereinfacht diesen Ablauf, indem es Kund*innen die Möglichkeit bietet in der App ein Foto ihres Zählers hochzuladen. Dieses wird dann 	
	ausgewertet und der Zählerstand  automatisch übermittelt.
	Wenn die App nicht benutzt wird, können die Zählerstände auch manuell auf der Website eingegeben werden.\\\\
	Die Administrierenden können die Daten der betreuten Kund*innen verwalten und die Zählerstände einsehen.
\section{Zielgruppen}
	Die App und Website sind für die Kund*innen des Energieanbieters konzipiert. Das Interface und die Verwendung sind einfach gehalten, sodass auch Nutzende mit wenig
	technischen Vorkenntnissen diese benutzen können.
	
	Auch Administrierende benötigen keine besonderen Vorkenntnisse, um die Webseitenansicht zur Verwaltung der Daten zu nutzen.
	Grundlegende Kenntnisse über die Benutzung von Computern genügen hierfür.
	
\section{Betriebsbedingungen}
	Die Persistierung der Datenbank erfolgt automatisch. Der Betrieb der Datenbanken sollte konstant von einem fachkundigem Mitarbeitenden überwacht werden.
	Die Anwendung erstellt keine Datensicherungen. Falls sie gewünscht sein sollten, so müssen diese manuell erfolgen.
	
