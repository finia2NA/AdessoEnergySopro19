Dieses Kapitel beschreibt die Rahmenbedingungen des Produkteinsatzes. Also die Anwendungsgebiete, die genauer bestimmen was der Zweck des Produktes ist, sowie die Zielgruppe, für die das Produkt entwickelt wird,
und die Betriebsbedingungen, also die Umstände unter denen eine Firma das Produkt benutzen und warten kann.

\section{Anwendungsgebiete}
	Um den aktuellen Zählerstand an den Energieanbieter zu schicken, mussten bisher die Stände abgelesen und per Post verschickt 		  
	werden. \\
	Das Produkt vereinfacht diesen Ablauf, indem es Kunden die Möglichkeit bietet in der App ein Foto ihres Zählers hochzuladen. Dieses wird dann 	
	ausgewertet und der Zählerstand  automatisch übermittelt.
	Wenn die App nicht benutzt wird, können die Zählerstände auch manuell auf der Website eingegeben werden.\\\\
	Die Administrierenden können die Daten der betreuten Kunden verwalten und die Zählerstände einsehen.
\section{Zielgruppen}
	Die App und Website sind für die Kunden des Energieanbieters konzipiert. Das Interface ist einfach gehalten, sodass auch Benutzer mit wenig technischen Vorkenntnissen es benutzen können.\\
	Auch Admins benötigen keine besonderen Vorkenntnisse, um die Webseitenansicht zur Verwaltung der Daten zu verwenden.
	Grundlegende Kenntnisse über die Benutzung von Computern genügen hierfür.
	
\section{Betriebsbedingungen}
	Die Persistierung der Datenbank erfolgt automatisch. Der Betrieb der Datenbank sollte konstant von einem fachkundigen Mitarbeitern überwacht werden.
	Die Anwendung erstellt keine Backups. Falls sie gewünscht sein sollten, so müssen diese manuell erfolgen.
	
