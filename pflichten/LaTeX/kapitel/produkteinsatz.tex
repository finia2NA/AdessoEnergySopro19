\begin{tcolorbox}
In diesem Kapitel werden die folgenden drei Punkte erläutert:
\begin{enumerate}
	\item \textit{Anwendungsgebiete:} Was ist der Zweck des Produkts?
	\item \textit{Zielgruppen:} Für welche Benutzer (oder auch Rollen) ist das Produkt bestimmt?
	Welche Qualifikationen brauchen die Personen?
	\item \textit{Betriebsbedingungen:} Automatische oder manuelle Datensicherung? 	Autonomer oder beobachtender Betrieb? 	
\end{enumerate}

\noindent Die einzelnen Teile des Produkteinsatzes werden üblicherweise als Fließtexte geschrieben.
\end{tcolorbox}

\section{Anwendungsgebiete}
	
	Das Produkt erlaubt es Kunden der Adesso AG ihren Strom- und Wasserzählerstand per Foto, bzw. als Texteingabe hochzuladen, 			sowie Zählerstände, die sie hochgeladen haben, einzusehen.
	Außerdem erlaubt es Administratoren den Datenbankinhalt zu verwalten und Supporttickets von Kunden zu bearbeiten.
	
	\section{Zielgruppen}
	
	Das Produkt ist für Kunden jeden Alters und Geschlechts mit nur geringerem oder besseren Computer/Smartphonekentnissen konzipiert.
	Außerdem werden Administratoren das Produkt nutzen, welche bessere Computerkentnisse haben sollten und in der Lage sein sollten Supporttickets bearbeten zu können.
	
	\section{Betriebsbedingungen}
	
	Der Betrieb wird nur unter der Beobachtung von trainierten Administratoren laufen können. Daten werden in einer Datenbank automatisch gesichert. Backups für erhöhte Datensicherheit müssen manuell erstellt werden.