\begin{tcolorbox}
Die Zielbestimmungen dienen dazu, die Ziele der Anforderungen nach Priorität zu sortieren. 
Es wird zwischen \textit{Muss}-, \textit{Soll-}, \textit{Kann-} und \textit{Abgrenzungskriterien} unterschieden, wobei weitere Einteilungen (z.B. nach Gerät oder Benutzer) innerhalb der Kategorien möglich sind.
%
\\\\
%
Die \textit{Musskriterien} umfassen alle Ziele und Funktionalitäten, die für einen Einsatz des entwickelten Produktes unabdingbar sind.
Sie müssen daher ohne Kompromisse implementiert werden.
Ein Wegfall eines einzelnen Musskriteriums würde das Produkt außer Betrieb setzen.
%
\\\\
%
\textit{Sollkriterien} (auch Wunschkriterien genannt) sind gewünschte Funktionen, die ebenfalls implementiert werden müssen, deren Wegfall auf Grund von unüblichen Umständen aber nicht den Einsatz des Produkts hindern würde.
%
\\\\
%
Die \textit{Kannkriterien} sind alle Ziele, die wünschenswert sind, aber nicht zwingend notwendige Funktionen darstellen. 
Oftmals werden diese nach Beendigung der höher priorisierten Kriterien umgesetzt.
%
\\\\
%
\textit{Abgrenzungskriterien} dienen dazu die Grenzen des Produkts zu definieren.
Es soll erkennbar sein, was explizit \textbf{nicht} umgesetzt wird, damit Kunden nichts Falsches erwarten und Ziele stets klar definiert bleiben.
%
\\\\
%
Für die Auflistung der Zielbestimmungen können Fließtexte oder auch Auflistungen mit ganzen Sätzen genutzt werden. 
\end{tcolorbox}

\section{Features}
Das folgende Kapitel behandelt alle erdachten Features und unterteilt diese in die Kategorien Muss-, Soll- oder Kann-Features. 
Die Muss-Features sind dabei essentiell für die Funktionalität der Software und haben höchste Priorität.
Soll-Features sind Erweiterungen der Grundfunktionen oder Verbesserungen der Muss-Features. Dabei ist die Grundfunktionalität der Software bereits durch die Muss-Features abgedeckt.
Wenn genug Zeit vorhanden ist, dann werden Funktionalitäten aus der Kategorie der Kann-Features implementiert. 
Diese stellen eine sinnvolle Erweiterung zum Projekt dar, sind aber im Gegensatz zu den Soll-Features nicht Teil der ursprünglichen Anforderungen.
\section{Muss-Features}
Die folgenden Features sind von uns als grundlegend eingestuft worden und umfassen die Kernfunktionalitäten von App und Anwendung.
\begin{itemize}
\item \textbf{History} \hfill \\
	In der App und auf der Website lassen sich die letzten Zählerstände für einen Account abrufen.
	Dabei lassen sich die letzten Zählerstände anzeigen.
\item \textbf{Scannen} \hfill \\
	In der App können Fotos aufgenommen und anschließend hochgeladen werden. 
	Wird von Azure ein Zähler auf dem Foto erkannt, dann wird der Zählerstand ausgelesen und an die Datenbank übermittelt.
	Außerdem wird das Foto gespeichert.
\item \textbf{Manuelles Eintragen} \hfill \\
	Auf der Website und in der App lassen sich Zählerart und Zählerstände manuell eintragen.
\item \textbf{Nutzerverwaltung} \hfill \\
	Mit Administrationsrechten können Kund*innendaten eingesehen, verändert oder neu ins System eingetragen werden.
	Dabei lassen sich Ergebnisse sortieren und filtern.
\end{itemize}

\section{Soll-Features}
Die Soll-Features sind Erweiterungen der Grundfunktionen oder stellen Verbesserungen bzw. Änderungen an diesen dar.
\begin{itemize}
\item \textbf{Kontaktformular} \hfill \\
	Nutzende können über ein Kontaktformular in der App oder auf der Website Anfragen schicken.
	In der Ansicht der Administrierenden können diese dann angesehen und bearbeitet werden. 
\item \textbf{Bild aus Galerie} \hfill \\
	Anstatt Bilder zum Auswerten des Zählerstandes in der App aufzunehmen, ist es auch möglich Bilder aus der Galerie des mobilen Endgerätes auszuwählen.
\item \textbf{Push-Nachrichten} \hfill \\
	Die App kann Benachrichtigungen schicken, um an das Eintragen von Zählerständen zu erinnern.
	Diese lassen sich bei Bedarf ausschalten.
\item \textbf{Mehrere Zähler}\hfill \\
	Einem Account können mehrere Zähler gleicher Art zugeordnet werden.
\item \textbf{Fehlerbehandlung} \hfill \\
	Falls ein falsches oder unleserliches Bild hochgeladen wird, erkennt die App dies und ein neues Foto wird angefordert.
\item \textbf{Fehler in Zahlen erkennen} \hfill \\
	Im Backend wird der hochgeladene Zählerstand auf Plausibilität überprüft. Dies geschieht durch Vergleiche mit den vorigen Zählerständen. 
	Bei nicht plausiblem Zählerstand wird der Benutzer zu einer erneuten Bestätigung aufgefordert. 
\end{itemize}

\section{Kann-Features}
Die Kann-Features sind weiter Funktionalitäten die nicht zum (erweiterten)-Grundumfang gehören und stellen sinnvolle Erweiterungen des Projekts dar.
\begin{itemize}
\item  \textbf{Darkmode}
	In der App wird eine Dunkle Benutzeroberfläche zur verfügung gestellt.
\item \textbf{Sprachen}
	Die Sprache der Website und der App lässt sich auf englisch umstellen.
\item \textbf{Diagramm}
	Die History wird um eine grafische Darstellung erweitert, die die Zählerstände eines Zeitabschnitts anzeigt.
\item \textbf{Statistiken}
	Mit Administrationsrechten können die durchschnittlichen oder spezifische Verbrauchs- und Zählerstatistiken eingesehen werden.
\item \textbf{App Erkennung}
	Wenn die Website auf einem Endgerät genutzt wird, welches die App installiert hat, dann wird dies erkannt und
	die Nutzung der App vorgeschlagen.
\end{itemize}

\section{Abgrenzungskriterien}

Nur das letzte Bild wird für User sichtbar sein, die letzten 10 für Admins.
Die App wird keine Verträge anzeigen oder ändern können.
Die Kunden werden nicht über die App bezahlen können.
Wir, das Entwicklerteam, können nicht für die Richtigkeit der von den Kunden hochgeladenen Zählerstände garantieren.
Ein Kunde wird nur genau ein Kundenkonto erhalten können.
Kunden werden sich nicht selbst registrieren können. Admins müssen für jeden Kunden ein Konto manuell erstellen.
Kunden können Bilder von Messständen nur über die App hochladen und nicht über die Website.

