\begin{tcolorbox}
Die Zielbestimmungen dienen dazu, die Ziele der Anforderungen nach Priorität zu sortieren. 
Es wird zwischen \textit{Muss}-, \textit{Soll-}, \textit{Kann-} und \textit{Abgrenzungskriterien} unterschieden, wobei weitere Einteilungen (z.B. nach Gerät oder Benutzer) innerhalb der Kategorien möglich sind.
%
\\\\
%
Die \textit{Musskriterien} umfassen alle Ziele und Funktionalitäten, die für einen Einsatz des entwickelten Produktes unabdingbar sind.
Sie müssen daher ohne Kompromisse implementiert werden.
Ein Wegfall eines einzelnen Musskriteriums würde das Produkt außer Betrieb setzen.
%
\\\\
%
\textit{Sollkriterien} (auch Wunschkriterien genannt) sind gewünschte Funktionen, die ebenfalls implementiert werden müssen, deren Wegfall auf Grund von unüblichen Umständen aber nicht den Einsatz des Produkts hindern würde.
%
\\\\
%
Die \textit{Kannkriterien} sind alle Ziele, die wünschenswert sind, aber nicht zwingend notwendige Funktionen darstellen. 
Oftmals werden diese nach Beendigung der höher priorisierten Kriterien umgesetzt.
%
\\\\
%
\textit{Abgrenzungskriterien} dienen dazu die Grenzen des Produkts zu definieren.
Es soll erkennbar sein, was explizit \textbf{nicht} umgesetzt wird, damit Kunden nichts Falsches erwarten und Ziele stets klar definiert bleiben.
%
\\\\
%
Für die Auflistung der Zielbestimmungen können Fließtexte oder auch Auflistungen mit ganzen Sätzen genutzt werden. 
\end{tcolorbox}