\begin{tcolorbox}
	In diesem Abschnitt werden Testfälle für die Anwendungsfälle der Produktfunktionen definiert.
	Diese sollen später ebenfalls als \textbf{reale Tests} implementiert werden.
	
	\autoref{fig:testfaelle-tabelle} stellt eine exemplarische Tabelle für die Beschreibung der zu testenden Anwendungsfälle dar. 
	Stil und Formatierung sind variabel.
	\end{tcolorbox}
	
	\begin{figure}[!h]
		\begin{center}
			\begin{tabularx}{\textwidth}{ p{.05\textwidth} | p{.25\textwidth} | p{.2\textwidth} | X }
				\textbf{Nr.} & \textbf{Anwendungsfall ID} & \textbf{Szenario} & \textbf{Erwartetes Verhalten} \\ \hline
				1 & 1 & Ein eingeloggter Benutzer lädt ein Bild erfolgreich hoch und bestätigt den angezeigten Wert. & Der Wert wurde in der Datenbank gespeichert.    \\ \hline
				2 & 1 & Ein eingeloggter Benutzer lädt ein Bild erfolgreich hoch und bestätigt den angezeigten Wert. & Der Wert wurde als neuer Eintrag in der History hinzugefügt.    \\ \hline
				3 & 1 & Der Zählerstand wird nicht auf dem Bild erkannt oder der angezeigte Wert wird nicht durch den Nutzer bestätigt. & Es wurde kein Wert in der Datenbank hinzugefügt. \\ \hline
				4 & 1 & Ein eingeloggter Benutzer lädt ein Bild erfolgreich hoch und bestätigt den angezeigten Wert.  & Das Datum des neuen Eintrags in der History widerspricht nicht der Aktualisierungswartezeit\\ \hline
				5 & 2 & Ein eingeloggter Benutzer trägt einen Wert manuell ein. & Der Wert wurde in die Datenbank gespeichert. \\ \hline
				6 & 2 & Ein eingeloggter Benutzer trägt einen Wert manuell ein. & Der Wert wurde als neuer Eintrag in der History hinzugefügt. \\ \hline
				7 & 2 & Ein eingeloggter Benutzer trägt einen Wert manuell ein. & Das Datum des neuen Eintrags in der History widerspricht nicht der Aktualisierungswartezeit \\ \hline
				8 & 4 & Der Admin hat erfolgreich einen neuen Zähler bei einem Nutzer hinzugefügt.   & Dem betroffenen Nutzer wird nun der hinzugefügte Zähler (eindeutig durch Zählernummer) aufgeführt. \\ \hline
			\end{tabularx}	
		\end{center}
		\caption{Beschreibung der Akteure}
		\label{fig:testfaelle-tabelle}
	\end{figure}
	