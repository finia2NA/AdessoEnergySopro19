\begin{tcolorbox}
	In diesem Abschnitt werden Testfälle für die Anwendungsfälle der Produktfunktionen definiert.
	Diese sollen später ebenfalls als \textbf{reale Tests} implementiert werden.
	
	\autoref{fig:testfaelle-tabelle} stellt eine exemplarische Tabelle für die Beschreibung der zu testenden Anwendungsfälle dar. 
	Stil und Formatierung sind variabel.
	\end{tcolorbox}
	
	\begin{figure}[!h]
		\begin{center}
			\begin{tabularx}{\textwidth}{ p{.05\textwidth} | p{.25\textwidth} | p{.2\textwidth} | X }
				\textbf{Nr.} & \textbf{Anwendungsfall ID} & \textbf{Szenario} & \textbf{Erwartetes Verhalten} \\ \hline
				1 & M1 & Ein eingeloggter Benutzer lädt ein Bild erfolgreich hoch und bestätigt den angezeigten Wert. & Der Wert wurde in der Datenbank gespeichert.    \\ \hline
				2 & M1 & Ein eingeloggter Benutzer lädt ein Bild erfolgreich hoch und bestätigt den angezeigten Wert. & Der Wert wurde als neuer Eintrag in der History hinzugefügt.    \\ \hline
				3 & M1 & Der Zählerstand wird nicht auf dem Bild erkannt. & Es wurde kein Wert der Datenbank hinzugefügt. \\ \hline
				4 & M1 & Die Zählernummer wird nicht auf dem Bild erkannt. & Es wurde kein Wert der Datenbank hinzugefügt. \\ \hline
				5 & M1 & Der Zählertyp wird nicht auf dem Bild erkannt. & Es wurde kein Wert der Datenbank hinzugefügt. \\ \hline
				6 & M1 &  Ein eingeloggter Benutzer lädt ein Bild erfolgreich hoch und lehnt den angezeigten Wert ab. & Es wurde kein Wert der Datenbank hinzugefügt. \\ \hline
				7 & M1 & Der Zählerstand, die Zählernummer und der Zählertyp wurden von Azure erkannt, aber der Zählerstand ist kürzer als das Format für diesen Zählertyp.  & Es wurde kein Wert der Datenbank hinzugefügt. \\ \hline
				8 & M1 & Der Zählerstand, die Zählernummer und der Zählertyp wurden von Azure erkannt, aber der Zählerstand ist länger als das Format für diesen Zählertyp.  & Es wurde kein Wert der Datenbank hinzugefügt. \\ \hline
		\end{tabularx}
	\end{center}
	\end{figure}

	\begin{figure}[!h]
		\begin{center}
			\begin{tabularx}{\textwidth}{ p{.05\textwidth} | p{.25\textwidth} | p{.2\textwidth} | X }
				\hline
				9 & M1 & Der Zählerstand, die Zählernummer und der Zählertyp wurden von Azure erkannt, aber die Zählernummer ist kürzer als ihr Format.  & Es wurde kein Wert der Datenbank hinzugefügt. \\ \hline
				10 & M1 & Der Zählerstand, die Zählernummer und der Zählertyp wurden von Azure erkannt, aber die Zählernummer ist länger als ihr Format.  & Es wurde kein Wert der Datenbank hinzugefügt. \\ \hline
				11 & M1 & Ein eingeloggter Benutzer lädt ein Bild erfolgreich hoch und bestätigt den angezeigten Wert.  & Das Datum des neuen Eintrags in der History widerspricht nicht aus vorigem Eintrag der Historie und Aktualisierungswartezeit. \\ \hline
				12 & M2 & Ein eingeloggter Benutzer trägt einen Wert manuell ein und bestätigt diesen. & Der Wert wurde in der Datenbank gespeichert. \\ \hline
				13 & M2 & Ein eingeloggter Benutzer trägt einen Wert manuell ein und bestätigt diesen. & Der Wert wurde als neuer Eintrag in der History hinzugefügt. \\ \hline
				14 & M2 & Ein eingeloggter Benutzer trägt einen Wert manuell ein und bestätigt diesen. & Das Datum des neuen Eintrags in der History widerspricht nicht der Kombination aus vorigem Eintrag der Historie und Aktualisierungswartezeit. \\ \hline
				15 & M2 & Ein eingeloggter Benutzer trägt einen Wert  manuell ein und bricht ab. & Der Wert wurde nicht in die Datenbank eingetragen. \\ \hline
				16 & M4 & Der Admin hat erfolgreich einen neuen Zähler bei einem Nutzer hinzugefügt.   & Dem betroffenen Nutzer wird nun der hinzugefügte Zähler (eindeutig durch Zählernummer) aufgeführt. \\ \hline
			\end{tabularx}	
		\end{center}
		\end{figure}
		
	\begin{figure}[!h]
		\begin{center}
			\begin{tabularx}{\textwidth}{ p{.05\textwidth} | p{.25\textwidth} | p{.2\textwidth} | X }
				\textbf{Nr.} & \textbf{Anwendungsfall ID} & \textbf{Szenario} & \textbf{Erwartetes Verhalten} \\ \hline
				7 & S1 & Ein eingeloggter Benutzer sendet erfolgreich ein Formular zur Kontaktaufnahme. & In der Datenbank wurde dieses Formular gespeichert. \\ \hline
                18 & S1 & Ein eingeloggter Benutzer öffnet das Formular zur Kontaktaufnahme, bricht die Aktion aber ab. & Es wurde kein neuer Eintrag in der Datenbank gespeichert. \\ \hline
                19 & S1 & Ein eingeloggter Benutzer versucht ein Formular zu senden, hat jedoch mindestens eines der Felder nicht ausgefüllt. &  Es wurde kein neuer Eintrag in der Datenbank gespeichert. \\ \hline
                20 & S2 & Ein eingeloggter Benutzer lädt ein Bild aus der Gallerie erfolgreich hoch und bestätigt den angezeigten Wert. & Der Wert wurde in der Datenbank gespeichert.    \\ \hline
                21 & S2 & Ein eingeloggter Benutzer lädt ein Bild aus der Gallerie erfolgreich hoch und bestätigt den angezeigten Wert. & Der Wert wurde als neuer Eintrag in der History hinzugefügt.    \\ \hline
                22 & S2 & Der Zählerstand wird nicht auf dem Bild erkannt. & Es wurde kein Wert der Datenbank hinzugefügt. \\ \hline
                23 & S2 & Die Zählernummer wird nicht auf dem Bild erkannt. & Es wurde kein Wert der Datenbank hinzugefügt. \\ \hline
                24 & S2 & Der Zählertyp wird nicht auf dem Bild erkannt. & Es wurde kein Wert der Datenbank hinzugefügt. \\ \hline
                25 & S2 & Ein eingeloggter Benutzer lädt ein Bild erfolgreich hoch und lehnt den angezeigten Wert ab. & Es wurde kein Wert der Datenbank hinzugefügt. \\ \hline
                26 & S2 & Der Zählerstand, die Zählernummer und der Zählertyp wurden von Azure erkannt, aber der Zählerstand ist kürzer als das Format für diesen Zählertyp.  & Es wurde kein Wert der Datenbank hinzugefügt. \\ \hline
                27 & S2 & Der Zählerstand, die Zählernummer und der Zählertyp wurden von Azure erkannt, aber der Zählerstand ist länger als das Format für diesen Zählertyp.  & Es wurde kein Wert der Datenbank hinzugefügt. \\ \hline
		\end{tabularx}
	\end{center}	
		
		\caption{Beschreibung der Testfälle}
		\label{fig:testfaelle-tabelle}
	\end{figure}
	