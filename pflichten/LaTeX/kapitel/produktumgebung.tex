\begin{tcolorbox}
In diesem Kapitel werden die folgenden Punkte erläutert. Eine jeweilige Unterteilung in Client und Server ist sinnvoll.
\begin{enumerate}
	\item \textit{Software:} Welche Software (Betriebssystem, Datenbanken, Webserver, externe Programme, etc.) ist auf den Zielsystemen für einen Betriebseinsatz erforderlich?
	\item \textit{Hardware:} Welche Hardware ist für den Produkteinsatz notwendig? Insbesondere Mindestanforderungen sind hier zu erwähnen.
	\item \textit{Orgware:} Umfasst organisatorische Anforderungen an die Produktumgebung, welche nicht unter die ersten beiden Kategorien fallen. 
	Dieser Punkt ist stark abhängig vom Projekt und kann auch nur weniger interessante Informationen, wie z.B. Zugang zum Internet umfassen.
	\item \textit{Produktschnittstellen:} Welche Schnittstellen werden zur Laufzeit von dem zu entwickelnden System genutzt (kurze textuelle Beschreibung)?
\end{enumerate}

\noindent Die einzelnen Abschnitte der Produktumgebung können als Fließtexte oder Absätze / Paragraphen mit ganzen Sätzen geschrieben werden.
\end{tcolorbox}

\section{Software}

\subsection{App}
Die App ist für alle Android Version ab 6 (Marshmallow) konzipiert und hält sich dabei an die Material Design Vorgaben von Google um sich so auch visuell ins System zu integrieren.

\subsection{Website}
Die Website wird auf Basis moderner Webstandardts entwickelt. Es wird nach Absprache mit den Kunden davon ausgegangen, dass ein moderner Browser (Firefox ab Version 67 oder Chrome ab Version 75) vorausgesetzt werden darf.

\subsection{Server} 
Serverseitig läuft unsere entwickelte Software in Docker Images. Es wird zusätzlich zur entwickelten Software voraussichtlich ein React als Front-End laufen.
Als Datenbank wird PostgreSQL benutzt.

\section{Hardware}
Die Software wird auf einem DualCore (3GHz) System mit 3GB RAM getestet. Im Deployment wird ein ähnliches oder besseres Setup empfohlen.\\

Auf Nutzerseite wird ein Computer oder Smartphone benötigt. 
\section{Orgware}

Das Produkt setzt eine stabile Internetverbindung vorraus. Es wird eine Domain empfohlen um den Zugriff auf den Server zu vereinfachen.

\section{Produktschnittstellen}
Es zwischen den einzelnen Modulen wird mittels einer REST-API kommuniziert.
